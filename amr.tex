\section{Adaptive Mesh Refinement} \label{sec_amr}
In this Section, we introduce Adaptive Mesh Refinement (AMR)
\cite{Jessee1998,Wang2010a,Ragusa2010}. The goal of AMR is to refine the mesh
where it is necessary while keeping the mesh as coarse as possible everywhere
else. It is common to have problems where the solution varies quickly on a
small part of the domain (e.g. boundary layers) where the mesh needs to be
very fined and is smooth on the rest of the domain where large cells can be
used. AMR allows to create such a mesh automatically.AMR requires an \emph{a
posteriori} error indicator to decide which cells should be refined. In this
work, the jump-based error indicator on element $K$ is given by \cite{Wang2010a}:
\begin{equation}
  \eta_K = \frac{\int_{\partial K} \ldb\phi_K\rdb^2}{\|\Phi_K\|_2^2},
  \label{error_indic}
\end{equation}  
is used. This error indicator is based on the on the jump of the scalar flux between
two cells. The larger the jump is the larger the error indicator. Given that
the solution of the diffusion equation is continuous, \cref{error_indic} is a
good error indicator. 

The way the adaptive mesh refinement is used in our code is the following:
\begin{enumerate}
  \item The solution is computed on a coarse mesh.
  \item The error indicator for each cell is computed.
  \item All the cells $K$ such that $\eta_K > \epsilon \max_{J} \eta_J$ where
    $\epsilon \in [0,1]$ are flagged for refinement.
  \item The flagged cells are refined.
  \item The solution on the coarse mesh is projected on the fine mesh.
  \item Go back to 1.
\end{enumerate}

When a discretization that does not allow polygonal cells is used, the
refinement of the mesh created hanging nodes. When the discretization allows
polygonal cells, hanging nodes are not necessary. Every time a hanging node
would be needed, one edge is added to 2 cells.
