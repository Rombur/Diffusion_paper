\section{Adaptive Mesh Refinement} \label{sec_amr}
In this Section, we introduce Adaptive Mesh Refinement (AMR)
\cite{Jessee1998,Wang2010a,Ragusa2010}. The goal of AMR is to refine the mesh
where it is necessary while keeping the mesh as coarse as possible everywhere
else. It is common to have problems with a quickly varying solution on a small
part of the domain (e.g. boundary layers) and a smooth solution on the rest of 
the domain. AMR allows to create automatically a mesh very fined where the
solution varies quickly while keeping the rest of the mesh coarse. AMR requires 
an \emph{a posteriori} error indicator to decide which cells should be refined. 
In this work, the error indicator on element $K$ is given by \cite{Wang2010a}:
\begin{equation}
  \eta_K = \frac{\int_{\partial K} \ldb\phi_K\rdb^2}{\|\Phi_K\|_2^2},
  \label{error_indic}
\end{equation}  
This error indicator is based on the jump of the scalar flux between
two cells. The larger the jump is the larger the error indicator. Given that
the solution of the diffusion equation is continuous, \cref{error_indic} is a
good error indicator. 

The adaptive mesh refinement used in our code works as follows:
\begin{enumerate}
  \item The solution is computed on a coarse mesh.
  \item The error indicator for each cell is computed.
  \item All the cells $K$ such that $\eta_K > \epsilon \max_{J} \eta_J$ where
    $\epsilon \in [0,1]$ are flagged for refinement.
  \item The flagged cells are refined.
  \item The solution on the coarse mesh is projected on the fine mesh.
  \item Go back to 1.
\end{enumerate}
Step 5 is not necessary but it permits to decrease the time needed to solve
the problem on the finer mesh since a very good initial guess is available to
the solver.

When a discretization that does not allow polygonal cells is employed, the
refinement of the mesh may create hanging nodes. Whereas if polygonal cells can
be used, hanging nodes are not necessary since every time a hanging node
would be needed, one edge is added to 2 cells.
