\section{Interior Penalty method} \label{sec_ip}
The discretization of the diffusion equation using discontinuous finite
elements is not as straightforward as it is when using continuous finite
elements. To discretize the diffusion equation with discontinuous finite elements, 
we apply the interior penalty method \cite{Kanschat2007}:
\begin{equation}
  -\bn D \bn \phi + \Sigma_a \phi = Q_0\ \textrm{ for } \br \in \mc{D},
\end{equation}
\begin{equation}
  \frac{1}{4}\phi - \frac{1}{2} D \partial_n \phi =0\ \textrm{ for } \br \in
  \partial \mc{D}^d,
\end{equation}
and
\begin{equation}
  -D \partial_n \phi = J^{inc}\ \textrm{ for } \br \in \partial \mc{D}^n,
\end{equation}
where $D$ is the diffusion coefficient, $\phi$ is the scalar flux, $\Sigma_a$
is the absorption cross section, $Q_0$ is a volumetric source, $J^{inc}$ is an
incoming current, $\mc{D}$ is the domain, $\partial \mc{D}^d$ is the boundary 
where Dirichlet conditions are applied, and $\partial \mc{D}^n$ is the boundary 
where Neumann conditions are applied.\\ 
After symmetrization, we get the SPD equation:
\begin{equation}
  a(\tilde{\phi},b) = l(b),
\end{equation}
where:
\begin{equation}
  \begin{split}
    a (\tilde{\phi},b) =& \(\Sigma_a \tilde{\phi},b\)_{\mc{D}} + 
    \(D\bn\tilde{\phi},\bn b\)_{\mc{D}} +
    (\kappa_e\ldb\tilde{\phi}\rdb,\ldb b \rdb)_{E_h^i}\\
    &+ \(\ldb\tilde{\phi}\rdb,\llb D\partial_n b \rrb\)_{E_h^i}+ \(\llb D
    \partial_n \tilde{\phi}\rrb,\ldb b \rdb\)_{E_h^i}\\
    &+ \(\kappa_e \tilde{\phi}, b\)_{\partial \mc{D}^d}
    -\frac{1}{2}\(\tilde{\phi},D\partial_n b\)_{\partial \mc{D}^d}
    -\frac{1}{2}\(D\partial_n\tilde{\phi},b\)_{\partial \mc{D}^d}
  \end{split}
\end{equation}
and: 
\begin{equation}
  l(b) = (Q_0,b)_{\mc{D}} + (J^{inc},b)_{\partial
  \mc{D}^r},
\end{equation}
with $\tilde{\phi}=(\phi_1 b_1,\hdots, \phi_N b_N)^T$, $b = (b_1,\hdots,b_N)$,
$b_i \in W_{\mc{D}}^h$ $\forall i \in (1,\hdots,N)$,
the mesh $\mc{T}_h$ is used to discretize $\mc{D}$ into nonoverlapping linear
elements $K$, such that the union of the elements fully covers $\mc{D}$. The
finite dimensional polynomial space is $W_{\mc{D}}^h = \{f \in L^2(\mc{D});
f|_K \in V_p(K), \forall K \in \mc{T}_h$\}, where $V_p(K)$ is the space of
polynomials of degree up to $p$ on element $K$; the set of interior is $E_h^i
= \cup _{K_1,K_2\in \mc{T}_h}(\partial K_1 \cap \partial K_2)$. We also
define:
\begin{align}
  \ldb \phi \rdb &= \phi^+ - \phi^-,\\
  \llb \phi \rrb &=  \frac{\phi^++\phi^-}{2},
\end{align}
with $\phi^{\pm} = \lim_{s\rightarrow 0^{\pm}} \phi(\br+s \bs{n}_e)$, where
$\bs{n}_e$ is the normal unit vector associated with an edge $e$. On the boundary, 
the normal vector has to be oriented outward whereas the orientation on an
interior edge is arbitrary.\\
The penalty parameter $\kappa_e$ is given:
\begin{equation}
  \kappa_e = \left\{
    \begin{aligned}
      &\frac{c(p^+)}{2}\frac{D^+}{h_{\bot}^+}+\frac{c(p^-)}{2}
      \frac{D^-}{h_{\bot}^-} & \textrm{ on interior edges, i.e., } e \in
      E_h^i,\\
      & c(p)\frac{D}{h_{\bot}} & \textrm{ on boundary edges, i.e., } e\in
      \partial \mc{D},
    \end{aligned}
    \right.
\end{equation}
where $c(p) =Cp(p+1)$, $C$ is a constant (we used $C=2$), $p$ is the
polynomial order,  and $h_{\bot}$ is the length of the cell in the direction 
orthogonal to edge $e$. The + and - symbols represent the values on either 
side of an edge. On triangular cells, $h_{\bot}$ is $\frac{2A}{L_e}$ where $A$
is the area of the triangle and $L_e$ is the length of the edge $e$. However
when the cells are not triangular, there is no simple simple way to compute
$h_{\bot}$. To simplify this, we assume that the polygonal cells are not too
far from being regular polygonal cells. In such cases, if the cell has an even
number of edges, the orthogonal length equals two times the apothem, i.e., two
times the segment between the midpoint of a side of the polygon and the center
of this polygon $\(\textrm{apothem}=2\times
\frac{\textrm{area}}{\textrm{perimeter}}\)$. If the cell has an odd number of
edges, the orthogonal length is given by the apothem plus the circumradius,
i.e., the radius of the circle circumscribed to the polygon
$\(\textrm{circumradius}=\sqrt{\frac{2\times\textrm{area}}{V \sin
\(\frac{2\pi}{N_V}\)}}\)$. Therefore, $h_{\bot}$ is given by
\Cref{ortho_length}.
\begin{table}[H]
  \begin{centering}
    \begin{tabular}{|c|c|c|c|c|}
      \hline
      Number of edges & 3 & 4 & $> 4$ and even & $>4$ and odd \\
      \hline
      $h_{\bot}$ & $2\times \frac{\textrm{area}}{L_e}$ &
      $\frac{\textrm{area}}{L_e}$ & $4\times
      \frac{\textrm{area}}{\textrm{perimeter}}$ &
      $2\times \frac{\textrm{area}}{\textrm{perimeter}} +
      \sqrt{\frac{2\times\textrm{area}}{V \sin\(\frac{2\pi}{N_V}\)}}$\\
      \hline
    \end{tabular}
    \caption{Orthogonal length of the cell for different cells.}
    \label{ortho_length}
  \end{centering}
\end{table}
