\section{Interior Penalty method} \label{sec_ip}
The discretization of the diffusion equation using discontinuous finite
elements is not as straightforward than it is when using continuous finite
elements. To discretize the diffusion with discontinuous finite elements, we 
apply the interior penalty method \cite{Kanschat2007}:
\begin{equation}
  -\bn D \bn \phi + \Sigma_a \phi = Q_0\ \textrm{ for } \br \in \mc{D},
\end{equation}
\begin{equation}
  \frac{1}{4}\phi - \frac{1}{2} D \partial_n \phi =0\ \textrm{ for } \br \in
  \partial \mc{D}^d,
\end{equation}
and
\begin{equation}
  -D \partial_n \phi = J^{inc}\ \textrm{ for } \br \in \partial \mc{D}^n,
\end{equation}
where $D$ is the diffusion coefficient, $\phi$ is the scalar flux, $\Sigma_a$
is the absorption cross section, $Q_0$ is a volumetric source, $J^{inc}$ is an
incoming current, $\mc{D}$ is the domain, $\partial \mc{D}^d$ is the boundary 
where Dirichlet conditions are applied, and $\partial \mc{D}^n$ is the boundary 
where Neumann conditions are applied. Therefore, we get:
\begin{equation}
  b_{IP}(\tilde{\phi},b) = l_{IP}(b),
\end{equation}
where:
\begin{equation}
  \begin{split}
    b_{IP} (\tilde{\phi},b) =& \(\Sigma_a \tilde{\phi},b\)_{\mc{D}} + 
    \(D\bn\tilde{\phi},\bn b\)_{\mc{D}} +
    (\kappa_e^{IP}\ldb\tilde{\phi}\rdb,\ldb b \rdb)_{E_h^i}\\
    &+ \(\ldb\tilde{\phi}\rdb,\llb D\partial_n b \rrb\)_{E_h^i}+ \(\llb D
    \partial_n \tilde{\phi}\rrb,\ldb b \rdb\)_{E_h^i}\\
    &+ \(\kappa_e^{IP} \tilde{\phi}, b\)_{\partial \mc{D}^d}
    -\frac{1}{2}\(\tilde{\phi},D\partial_n b\)_{\partial \mc{D}^d}
    -\frac{1}{2}\(D\partial_n\tilde{\phi},b\)_{\partial \mc{D}^d}
  \end{split}
\end{equation}
and:
\begin{equation}
  l_{IP}(b) = (Q_0,b)_{\mc{D}} + (J^{inc},b)_{\partial
  \mc{D}^r},
\end{equation}
with $\tilde{\phi}=(\phi_1 b_1,\hdots, \phi_N b_N)^T$, $b = (b_1,\hdots,b_N)$,
$b_i \in W_{\mc{D}}^h$ $\forall i \in (1,\hdots,N)$,
the mesh $\mc{T}_h$ is used to discretize $\mc{D}$ into nonoverlapping linear
elements $K$, such that the union of the elements fully covers $\mc{D}$. The
finite dimensional polynomial space is $W_{\mc{D}}^h = \{f \in L^2(\mc{D});
f|_K \in V_p(K), \forall K \in \mc{T}_h$, where $V_p(K)$ is the space of
polynomials of degree up to $p$ on element $K$; the set of interior is $E_h^i
= \cup _{K_1,K_2\in \mc{T}_h}(\partial K_1 \cap \partial K_2)$. We also
define:
\begin{align}
  \ldb \phi \rdb &= \phi^+ - \phi^-,\\
  \llb \phi \rrb &=  \frac{\phi^++\phi^-}{2}.
\end{align}
The penalty parameter $\kappa_e^{IP}$ is given:
\begin{equation}
  \kappa_e^{IP} = \left\{
    \begin{aligned}
      &\frac{c(p^+)}{2}\frac{D^+}{h_{\bot}^+}+\frac{c(p^-)}{2}
      \frac{D^-}{h_{\bot}^-} & \textrm{ on interior edges, i.e., } e \in
      E_h^i,\\
      & c(p)\frac{D}{h_{\bot}} & \textrm{ on boundary edges, i.e., } e\in
      \partial \mc{D}^n,
    \end{aligned}
    \right.
\end{equation}
where $c(p) =Cp(p+1)$, $C$ is a constant (we used $C=2$), $p$ is the
polynomial order, $D$ is the diffusion coefficient, and $h_{\bot}$ is the
length of the cell in the direction orthogonal to edge $e$.  The + and -
symbols represent the values on either side of an edge. With the penalty
coefficient $\kappa_e^{IP}$, the IP bilinear form is symmetric positive
definite.
