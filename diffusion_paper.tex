\documentclass[letterpaper]{article}
\usepackage{amsmath}
\usepackage{array}
\usepackage{color}
\usepackage{graphicx}
\usepackage{float} % utiliser H pour forcer a mettre l'image ou on veut
\usepackage{lscape} % utilisation du mode paysage
\usepackage{mathbbol} % permet d'avoir le vrai symbol pour les reels grace a mathbb
\usepackage{enumerate} % permet d'utiliser enumerate
\usepackage{moreverb} % permet d'utiliser verbatimtab : conservation la tabulation
\usepackage{stmaryrd} % permet d'utiliser \llbrackedt et \rrbracket : double crochet
\usepackage[noabbrev]{cleveref} % permet d'utiliser cref and Cref
\usepackage{caption} % permet d'utiliser subcaption
\usepackage{subcaption} % permet d'utiliser subfigure, subtable, etc
\usepackage[margin=1.in]{geometry}

\newcommand\bn{\boldsymbol{\nabla}}
\newcommand\bo{\boldsymbol{\Omega}}
\newcommand\br{\mathbf{r}}
\newcommand\la{\left\langle}
\newcommand\ra{\right\rangle}
\newcommand\bs{\boldsymbol}
\newcommand\red{\textcolor{red}}
\newcommand\blue{\textcolor{blue}}
\newcommand\ldb{\{\!\!\{}
\newcommand\rdb{\}\!\!\}}
\newcommand\llb{\llbracket}
\newcommand\rrb{\rrbracket}
\newcommand\mc{\mathcal}

\renewcommand{\(}{\left(}
\renewcommand{\)}{\right)}
\renewcommand{\[}{\left[}
\renewcommand{\]}{\right]}

\usepackage{setspace}
%\doublespacing

\begin{document}
%\title{Discontinuous Finite Element Solution for Diffusion Equations on Arbitrary Polygonal Meshes}
\title{Discontinuous Finite Element Solution of the Diffusion Equation on Arbitrary Polygonal Meshes}
\author{} 
\date{}
\maketitle

%%%%%%%%%%%%%%%%%%%%%%%%%%%%%%%%%%%%%%%%%%%%%%%%%%%%%%%%%%%%%%%%%%%%%%%%%%%%%%%%%%%%%%%%
\section{Introduction} \label{sec_intro}
%%%%%%%%%%%%%%%%%%%%%%%%%%%%%%%%%%%%%%%%%%%%%%%%%%%%%%%%%%%%%%%%%%%%%%%%%%%%%%%%%%%%%%%%

This paper deals with a discontinuous finite element spatial discretizations of the radiation 
diffusion equation on arbitrary polygonal grids, with and without adaptive mesh refinement. 
Radiation diffusion is an asymptotic limit of the radiation transport equation and can be 
written in the following form:
\begin{equation} \label{eq:radiation_diffusion}
- \div  D(\vr) \grad E(\vr) + \sigma_a(\vr) E(\vr) = Q(\vr) ,
\end{equation}
where $E$ is the radiation energy intensity, $D$ is a diffusion coefficient, $\sigma_a$ is 
an opacity coefficient, and $Q$ is the source.

Several spatial discretizations have been proposed to solve \eqt{eq:radiation_diffusion} on
arbitrary polygons (2D) and polyhedra (3D) \cite{Wachspress,Kuznetsov2004,Palmer2005,Brezzi2005,
LipnikovShashkovSvyatskiy2006,BaileyAdams2008}. 

Wachspress \cite{Wachspress} developed a family of rational polynomial functions that can be employed
as basis functions in a finite element method on polygonal/polyhedral grids. This yields
symmetric positive-definite (SPD) matrices but (i) the finite element integrals must be carried out 
numerically and (ii) the Jacobian of the transformation becomes zero on degenerate cells 
(such as the ones shown on \fig{fig:amr_schematics}). 
%
%Morel et al. \cite{MorelDendyHallWhite1992} introduced a cell-centered finite volume scheme 
%for arbitrary quadrilateral meshes. Their scheme was second-order accurate and yielded back a 
%standard five-point stencil on orthogonal grids, but the diffusion operator was asymmetric 
%and cell-edge unknowns were added in addition to cell-center unknowns.
%%
Palmer \cite{Palmer2005,PalmerLLNL} proposed a node-based finite volume method 
that enforces particle balance over dual cells, where a dual cell is defined as 
the union of all corners surrounding a given vertex $p$ and where  a corner 
is a quadrilateral defined by vertex $p$, the cell center, and the midpoint
of the edges that contain vertex $p$. On a triangular grid, Palmer's scheme is equivalent 
to linear continuous finite elements with ``mass-matrix lumping''. The method is 
second-order accurate but the discretization of the diffusion equation using Palmer's method 
does not result in SPD matrices.
%
Mimetic finite difference methods create discrete analog of vector and tensor
calculus in order to accurately approximate the original differential operators;
see, e.g., \cite{HymanMorelShashkovSteinberg2002}.
Mimetic methods preserve important properties of the differential operators such 
as symmetry, positivity, monotonicity, asymptotic limits, and identities pertaining 
to tensor and vector calculus. Mimetic methods can also be viewed as mixed hybrid 
finite element formulations with specific spatial quadratures.  
In addition to quadrilateral and hexahedral meshes (see, e.g., 
\cite{MorelRobertsShashkov1998,MorelHallShashkov2001}, mimetic finite difference 
methods have recently been applied to the diffusion equation on arbitrary polygonal 
grids \cite{Kuznetsov2004,Brezzi2005,LipnikovShashkovSvyatskiy2006,LipnikovShashkov2010}.
%conformal quadrilateral \cite{HymanShashkovSteinberg1997, ShashkovSteinberg1996,MorelRobertsShashkov1998} 
%and hexahedral \cite{MorelHallShashkov2001} grids, locally refined grids \cite{LipnikovMorelShashkov2004}, and
%arbitrary polygonal grids \cote{Kuznetsov2004,Brezzi2005,LipnikovShashkovSvyatskiy2006,LipnikovShashkov2010}.
%
%, scalar and vector inner products, such as: ,
%conservation laws, symmetry preservation, solution positivity and
%monotonicity, and asymptotic limits (e.g., diffusion limit), on polygonal and
%polyhedral meshes. The most important part of MFD is the definition of a scalar
%product which satisfies stability and consistency some conditions
%\cite{Brezzi2005}. However, this scalar product is not unique and therefore,
%multiple MFD methods exists. MFD is efficient even on concave polygons
%\cite{Kuznetsov2004}. MFD methods are related to mixed finite elements.








In this paper, we are interested in solving the diffusion equation on a
polygonal mesh. First, we want to point the usefulness of using polygonal
cells to discretize the domain of a problem. Such cell type presents a big 
advantage over traditional cells type (triangles and rectangles): polygonal 
cells allow for meshing flexibility. Boundary layer meshes can easily be set 
up, polygonal meshes can be generated from triangular meshes, and polygons 
can be included locally in existing meshes to improve mesh quality. Existing 
meshing tools such as MSTK \cite{mstk} and the Computational Geometry Algorithms 
Library \cite{cgal} may be employed to process polygonal meshes. For 
instance, the radiation transport code PDT and the CFD codes Fluent and OpenFoam 
offer polygonal mesh and solver capabilities. The following features of polygonal
cells are noteworthy:
\begin{description}
  \item[Optimal partition of the space minimizing boundary/interior ratio]
  \item[Reduced number of unknowns:] To illustrate this, we assume
    one unknown per vertex in every cell, which is standard for linear discontinuous
    finite element transport discretizations that perform well in the thick
    diffusive regime. In the 2D hexagonal example of \Cref{fig_hex_vs_tri},
    the number of unknowns would be six (one unknown per vertex). Using
    triangular cells, the same hexagon would have to be split into four
    triangles at least (thus 12 unknowns) or possibly six triangles to
    preserve symmetry (thus 18 unknowns in that case). Similarly, using
    quadrilateral cells, the hexagon would be bisected into two quadrilaterals
    at least (8 unknowns), but divisions into three of four quadrilaterals are
    also possible (thus, 12 or 16 unknowns).
    \begin{figure}[H]
      \centering
      \includegraphics[width=5cm]{hex_tri_cells}
      \caption{Hexagonal cell versus triangle cells}
      \label{fig_hex_vs_tri}
    \end{figure}
  \item[Transition elements and Adaptive Mesh Refinement:] Solvers based on
    arbitrary polyhedral cells can easily handle cells with various number of
    edges. This can be particularly useful for simulations
    with Adaptive Mesh Refinement (AMR) \cite{Jessee1998,Baker2002,Wang2010a}, 
    without having to deal with the implementation of data structures to handle 
    hanging nodes \cite{Solin2008,Bangerth2007,Arnold2000}. On \Cref{fig_amr_cells}, 
    the left cell is a pentagon whereas the two cells on the right are 
    quadrilaterals. A method based on a piecewise linear discretization can
    handle locally adapted meshes without any special treatment or further
    approximation of the coupling between cells.
    \begin{figure}[H]
      \centering
      \includegraphics[width=5cm]{amr}
      \caption{AMR mesh}
      \label{fig_amr_cells}
    \end{figure}
\end{description}
Several discretization methods have been developed for arbitrary polygonal
meshes: Palmer's method \cite{Palmer2001}, mimetic finite differences
\cite{Lipnikov2004,Hyman2002,Kuznetsov2004,Brezzi2005},
Wachspress' rationale finite element \cite{Wachspress1975},
CFEM-based DFEM \cite{Warsa2008}, PWLC \cite{Bailey2008a}, PWLD 
\cite{Stone2003,Bailey2008,Bailey2008a}, and PWBLD \cite{Bailey2011}. In this
research, we focus on using PWLD to discretize the diffusion equation. The
PWLD discretization employs discontinuous finite elements and has been used to
discretize the transport equation. Using it to discretize the diffusion
equation is an important step in order to create a Diffusion Synthetic
Acceleration scheme \cite{Adams2002,Wang2010}.
In \Cref{sec_review}, we review different discretizations that can be used on
polygonal cells to discretize the diffusion equation. In \Cref{sec_ip}, we
use the PWLD finite elements to discretize the diffusion
equation. In \Cref{sec_amr}, we introduce the Adaptive Mesh Refinement
technique (AMR). In \Cref{sec_results}, we show some numerical results. We finish
in \Cref{sec_conc} by giving our conclusions.

\section{Review of discretizations for polygonal meshes} \label{sec_review}
In this section, we review discretizations that can be used on polygonal
meshes at the exception of the PWLD discretization which will be introduce in
the next Section.
\subsection{Palmer's method}
Palmer's method is a node- or point-based method \cite{Palmer2001}. The method
is second order. This discretization uses a finite volume approach: the
particle balance is enforced by integrating the diffusion equation over a
control volume. The control volume is the union of all corners surrounding the
specified point. A corner is a quadrilateral with vertices at the point $p$,
the cell-center $c$, and the midpoint of each of the edges $e$ that contain
the point $p$. On a triangular grid, the scheme is equivalent to linear
continuous finite elements with ``mass-matrix lumping. This method works for
any polygonal cell even concave polygon. The main disadvantage of the method
is that the system is not symmetric.
\subsection{Mimetic finite difference}
Mimetic methods, like mimetic finite difference, mimic properties of
mathematical and physical models, such as: tensor and vector calculus,
conservation laws, symmetry preservation, solution positivity and
monotonicity, and asymptotic limits (e.g., diffusion limit), on polygonal and
polyhedral meshes. The most important part of MFD is the definition of scalar
product which should satisfy stability and consistency conditions
\cite{Brezzi2005}. However, this scalar product is not unique. MFD is
efficient event on concave polygons \cite{Kuznetsov2004}. MFD method is
related to mixed finite elements.

%\red{An important (mimetic) property of operators div and $K\bn$ is expressed
%  by integration by part formula:
%  \begin{equation}
%    \int_{\bo} K^{-1}\bs{u} \cdot (K\bn p)dV = - \int_{\bo} (div\ \bs{u}) p\ dV
%  \end{equation}
%  which holds for any $p \in H_0^1(\bo)$ and $\bs{u}\in H_{div}(\bo)$.\\
%  The MFD method mimics this property by replacing integrals and operators by
%  their discrete counterparts:
%  \begin{equation}
%    [\bs{U},GRAD\ P]_X = -[DIV\ \bs{U},P]_Q
%  \end{equation}
%which holds for any vectors of degrees of freedom $P$ and $\bs{U}$. MFD
%connected to mixed finite elements. Works with mixed cells.}
%\red{Mimetic numerical methods mimic crucial properties of mathematical and
%  physical models on arbitrary on polygonal and polyhedral meshes:
%  \begin{itemize}
%    \item tensor and vector calculus
%    \item conservation laws
%    \item symmetry preservation
%    \item solution positivity and monotonicity
%    \item asymptotic limits (e.g., diffusion limit in radiation transport
%      models)
%  \end{itemize}}
%\red{\cite{Lipnikov2004} only quadrilaterals with at most one hanging node per edge. 
%  SO can be applied to many forms of the diffusion equation, but here we consider
%the first-order form of the diffusion equation. Use local SO.
%\begin{itemize}
%  \item there are two different support-operators methods, they both
%        give the same cell-center intensity solutions:
%    \begin{itemize}
%      \item one has intensity unknowns located only at cell centers: the
%        diffusion matrix is dense.
%      \item one has intensity unknowns located at both cell centers and face
%        centers: the diffusion matrix is space. The cell-center unknowns are
%        locally eliminated, resulting in a reduced system consisting only of
%        the face-center unknowns. The method is called the local SO method.
%    \end{itemize}
%  \item local SO is composed of two steps:
%    \begin{itemize}
%      \item consider each cell in the domain as a independent domain and
%        generate an independent discretization for each cell (identical to
%        what is done for standard quadrilateral meshes)
%      \item obtain a global discretization by imposing continuity of the
%        intensity and continuity of the normal component of the flux across
%        cell interfaces. 
%    \end{itemize}
%\end{itemize}
%\cite{Hyman2002} the mimetic FDM are based on discrete of first-order
%coordinate-invariant operators, it is natural to write the diffusion equation
%as a system of first order equations.
%\cite{Kuznetsov2004} works for any polygons. Used in geophysics $->$ the equation is
%split in two first-order equations.
%\begin{itemize}
%  \item exact for linear solution
%  \item SPD
%\end{itemize}
%\cite{Brezzi2005} the key element of the MFD method is the scalar product in the
%space of discrete velocities which should satisfy the stability assumption and
%the consistency assumption. It turns out that such a scalar product is not
%unique.
%\begin{enumerate}
%  \item specify the degrees of freedom for the primary variables
%  \item define suitable scalar products in the discrete spaces
%  \item discretize the divergence operator
%  \item define the discrete flux operator
%\end{enumerate}}
\subsection{Wachspress' rationale finite element}
The disadvantages of this method is that it works only for convex cells and
the basis function integrals must be done numerically \cite{Bailey2008a}.
Before we introduce the basis function for a quadrilateral cell, we define:
\begin{equation}
  P_{1,2}(x,y) = (y-y_1) (x_2-x_1)-(x-x_1)(y_2-y_1).
\end{equation}  
The basis functions are given by:
\begin{align}
  &b_0(x,y) = k_0 P_{2,3}(x,y)P_{1,2}(x,y)/P_{4,5}(x,y),\\
  &b_1(x,y) = k_1 P_{0,3}(x,y)P_{2,3}(x,y)/P_{4,5}(x,y),\\
  &b_2(x,y) = k_2 P_{0,1}(x,y)P_{3,0}(x,y)/P_{4,5}(x,y),\\
  &b_3(x,y) = k_3 P_{1,2}(x,y)P_{0,1}(x,y)/P_{4,5}(x,y),
\end{align}
where $k_i$ are chose such that $b_i(x_i,y_i)=1$. $(x_4,y_4)$ and $(x_5,y_5)$
are the two points which are the intersection of the lines created by the edges of the
quadrilaterals (see \Cref{fig_quadrilateral}).
\begin{figure}[H]
  \centering
  \includegraphics[width=5cm]{quadrilateral}
  \caption{Quadrilateral cell}
  \label{fig_quadrilateral}
\end{figure}
These two points uniquely defines a linear polynomial. It
should be noted that the line defined does not pass through the quadrilateral
and thus, there is no risk that the denominator equals zero (this is only true
for convex quadrilateral). When the quadrilateral tends to a parallelogram,
the points defining $P_{4,5}$ are moved to the infinity and the value of
$P_{4,5}(x,y)$ with in $(x,y)$ in the cell tends to a constant. By definition
this constant is chosen to be 1 for a parallelogram. For a trapezoid,
$P_{4,5}$ is the line defined by the intersection of the lines generated by
the two edges which are not parallel and parallel to the two parallel edges of
the trapezoid. For arbitrary polygons, a non-trivial generalization is
necessary \cite{Wachspress1975}.
%\red{\cite{Bailey2008a} he disadvantage is that the basis function integrals must
%be done numerically.
%\red{\cite{Wachspress1975} $(r;q)_2|_3$ is the value at point 3 of a quadratic
%  function containing points $r$ and $q$. This value depends on supplementary
%  data which defines the quadratic function $(r;q)_2$. Similarly
%  $[(1;2)_2(3;4)/(1;5)]|_8$ is the value of the indicated rational function
%  at point 8. For quadrilaterals: we look for a function like:
%  \begin{equation}
%    w_1(x,y) = \frac{Q_1}{(2;3)(3;4)}|_1 \frac{(2;3)(3;4)}{Q_1}
%  \end{equation}
%  We therefore seek a linear from $Q_1$ such that:
%  \begin{itemize}
%    \item $Q_1\neq 0$ within the quadrilateral
%    \item $\frac{(2;3)(3;4}{Q_1}$ is linear on both $(4;1)$ and $(1;2)$
%  \end{itemize}
%  The first property has far-reaching consequences: we must broaden our vision
%  and look outside the quadrilateral. We observe that all the linear forms
%  appearing in the triangle and parallelogram wedges were determined by the
%  sides of these figures. We shall soon see that the quadrilateral itself
%  reaches out to give us the desired linear form. First, we need the following
%  lemma: \emph{If three lines intersect at a point then the ratio of linear
%  forms which vanish on any two of these lines is constant on the third line.}
%  We choose $Q_1$ so that $(2;3)/Q_1$ is constant on side $(4;1)$ and so that
%  $(3;4)/Q_1$ is constant on side $(4;1)$ and so that $(3;4)/Q_1$ is constant
%  on side $(1;2)$. By the lemma, the first requirement is met if lines
%  $(2;3)$, $(4;1)$, and $Q_1$ have a common point of intersection and the
%  second requirement is met if lines $(3;4)$, $(1;2)$, and $Q_1$ have a common
%  point of intersection. If we define point 5 as the intersection point of
%  lines $(2;3)$ and $(1;4)$ and point 6 as the intersection of $(1;2)$ and
%  $(3;4)$, we find that $Q_1(x,y) = (5;6)$ is the unique line which meets both
%  requirements. For any convex quadrilateral, line $Q_1$ has no point in the
%  quadrilateral. Consistent candidates for all four wedges are:
%  \begin{subequations}
%    \begin{align}
%      &W_1(x,y) = k_1 (2;3)(3;4)/Q_1(x,y)\\
%      &W_2(x,y) = k_2 (3;4)(4;1)/Q_1(x,y)\\
%      &W_3(x,y) = k_3 (4;1)(1;2)/Q_1(x,y)\\
%      &W_4(x,y) = k_4 (1;2)(2;3)/Q_1(x,y)
%    \end{align}
%  \end{subequations}
%  The $k_i$ are chose so that $W_i(x_i,y_i)=1$. As the quadrilateral is
%  deformed into a parallelogram, the exterior diagonal moves to infinity and
%  the associated linear from moves to infinity and the associated linear form
%  becomes more nearly constant within the quadrilateral. We therefore let
%  $Q_1(x,y)=1$ for a parallelogram. For a trapezoid, the exterior diagonal is
%  parallel to the parallel sides and passes through the intersection point of
%  the others two sides. We note that the exterior diagonal is uniquely defined
%  as the lined that intersects the sides of the quadrilateral at all the
%  exterior intersection points of these sides and at no other points.
%A non-trivial generalization is necessary for polygons.} 
\subsection{CFEM-based DFEM}
This method is very similar to the PWLD discretization. The scheme is second
order \cite{Warsa2008}. First, the polygonal cell is divided into simplexes
that we will call sub-call. The simplexes are formed by filling the polygons
with triangles without adding point (type 0) or by the vertices of the
cell and the center of the cell (type 1) cf PWLD. Linear DFEM are built on the 
triangles. If type 1 is used, CFEM-based DFEM is similar to PWLD but whereas
PWLD eliminates the center unknown by setting it equal to a weighted average
of the values at the vertexes, CFEM-based DFEM keeps the additional degree of
freedom as an unknown. Therefore, additional degrees of freedom can be added
to improve the numerical properties. The mass matrix $\bs{M}$ for a given cell
composed of $N$ sub-cells is assembled:
\begin{equation}
  \sum_{n=1}^N \bs{M}_{t_n(u),t_n(v)} + \bs{M}_{n,u,v}, \textrm{ for }
  u=1,2,3, v=1,2,3
\end{equation}
\blue{Loop over the sub-cell, and for each cell $n$:
  \begin{equation}
    \bs{M}_{t_n(u),t_n(v)} = \bs{M}_{n,u,v}
  \end{equation}
}
where $\bs{M}_n$ is the mass matrix of the $n$ sub-cell and $t_n$ is indexing
function which maps the index of the vertices of the sub-cells to the index of
the vertices in the cell. The streaming matrix is given by:
\begin{equation}
  \sum_{n=1}^N \bs{L}_{t_n(u),t_n(v)} + \bs{L}_{n,u,v}, \textrm{ for }
  u=1,2,3, v=1,2,3
\end{equation}
and the matrix $\bs{N}_{i,j}^{(k)} = \bs{n}_k \int_{\partial E_k} B_i(\br)
B_j(\br) d\br$ is given by:
\begin{equation}
  \bs{N}_{t_n(u),t_n(v)}^{(f_n(l))} = \left\{
    \begin{aligned}
      & \bs{N}_{n,u,v}^{(l)}, & \textrm{ if }f_n(l)\neq 0\\
      & 0, & \textrm{ otherwise }
    \end{aligned}
    \right.
    \textrm{ for } l=1,2,3,u=1,2,3,v=1,2,3
\end{equation}
where $f_n(k)$ is an indexing function to map face $k$ of the subelement $n$,
for $n=1,\hdots,N,$ to the faces of the polygon.
\subsection{PWBLD}
PWBLD stands for PieceWise Bi-Linear Discontinuous finite elements. This 
discretization is very similar to the PWLD discretization that we explain
in detail in the next Section. When using PWBLD, the convex polygonal cell is 
divided on quadrilateral instead of triangular for PWLD \cite{Bailey2011}. On 
these quadrilaterals, the basis functions are equivalent to bi-linear functions 
on quadrilaterals. If the cell is quadrilateral, PWBLD finite elements are 
identical to bi-linear discontinuous finite elements.
\subsection{PWLC}
PWLC stands for PieceWise Linear Continuous finite elements. It is a second
order method \cite{Bailey2008}. Unlike Palmer's method, the diffusion equation
discretized by PWLD produces a SPD matrix. The PWL function centered at vertex
$j$ is given by:
\begin{equation}
  b_j(\br) = t_j(\br) + \alpha_{c,j} t_c(\br)
\end{equation}
where the $t_j$ functions are standard linear functions defined triangle by
triangle. For instance, $t_ j$ equals 1 at the $j^{th}$ vertex of the cell and
0 at the $j-1^{th}$ and $j+1^{th}$ vertices. $t_c$ is the hat function which
is equal to one at point $c=(x_c,y_c)$ and 0 at all the vertices, with $x_c =
\sum_{i=0}^{N_v} \alpha_{c,i} x_i$, $y_c = \sum_{i=0}^{N_v} \alpha_{c,i} y_i$,
and $\sum_{i=0}^{N_V} \alpha_{c,i} = 1$. PWLC is a second order method. When
this discretization is applied to the diffusion equation, the system obtained
is SPD.
\subsection{PWLD}
PWLD uses similar basis functions than PWLC but they are defined only on a
element.

\section{Interior Penalty method} \label{sec_ip}
The discretization of the diffusion equation using discontinuous finite
elements is not as straightforward than it is when using continuous finite
elements. To discretize the diffusion with discontinuous finite elements, we 
apply the interior penalty method \cite{Kanschat2007}:
\begin{equation}
  -\bn D \bn \phi + \Sigma_a \phi = Q_0\ \textrm{ for } \br \in \mc{D},
\end{equation}
\begin{equation}
  \frac{1}{4}\phi - \frac{1}{2} D \partial_n \phi =0\ \textrm{ for } \br \in
  \partial \mc{D}^d,
\end{equation}
and
\begin{equation}
  -D \partial_n \phi = J^{inc}\ \textrm{ for } \br \in \partial \mc{D}^n,
\end{equation}
where $D$ is the diffusion coefficient, $\phi$ is the scalar flux, $\Sigma_a$
is the absorption cross section, $Q_0$ is a volumetric source, $J^{inc}$ is an
incoming current, $\mc{D}$ is the domain, $\partial \mc{D}^d$ is the boundary 
where Dirichlet conditions are applied, and $\partial \mc{D}^n$ is the boundary 
where Neumann conditions are applied. Therefore, we get:
\begin{equation}
  b_{IP}(\tilde{\phi},b) = l_{IP}(b),
\end{equation}
where:
\begin{equation}
  \begin{split}
    b_{IP} (\tilde{\phi},b) =& \(\Sigma_a \tilde{\phi},b\)_{\mc{D}} + 
    \(D\bn\tilde{\phi},\bn b\)_{\mc{D}} +
    (\kappa_e^{IP}\ldb\tilde{\phi}\rdb,\ldb b \rdb)_{E_h^i}\\
    &+ \(\ldb\tilde{\phi}\rdb,\llb D\partial_n b \rrb\)_{E_h^i}+ \(\llb D
    \partial_n \tilde{\phi}\rrb,\ldb b \rdb\)_{E_h^i}\\
    &+ \(\kappa_e^{IP} \tilde{\phi}, b\)_{\partial \mc{D}^d}
    -\frac{1}{2}\(\tilde{\phi},D\partial_n b\)_{\partial \mc{D}^d}
    -\frac{1}{2}\(D\partial_n\tilde{\phi},b\)_{\partial \mc{D}^d}
  \end{split}
\end{equation}
and:
\begin{equation}
  l_{IP}(b) = (Q_0,b)_{\mc{D}} + (J^{inc},b)_{\partial
  \mc{D}^r},
\end{equation}
with $\tilde{\phi}=(\phi_1 b_1,\hdots, \phi_N b_N)^T$, $b = (b_1,\hdots,b_N)$,
$b_i \in W_{\mc{D}}^h$ $\forall i \in (1,\hdots,N)$,
the mesh $\mc{T}_h$ is used to discretize $\mc{D}$ into nonoverlapping linear
elements $K$, such that the union of the elements fully covers $\mc{D}$. The
finite dimensional polynomial space is $W_{\mc{D}}^h = \{f \in L^2(\mc{D});
f|_K \in V_p(K), \forall K \in \mc{T}_h$, where $V_p(K)$ is the space of
polynomials of degree up to $p$ on element $K$; the set of interior is $E_h^i
= \cup _{K_1,K_2\in \mc{T}_h}(\partial K_1 \cap \partial K_2)$. We also
define:
\begin{align}
  \ldb \phi \rdb &= \phi^+ - \phi^-,\\
  \llb \phi \rrb &=  \frac{\phi^++\phi^-}{2}.
\end{align}
The penalty parameter $\kappa_e^{IP}$ is given:
\begin{equation}
  \kappa_e^{IP} = \left\{
    \begin{aligned}
      &\frac{c(p^+)}{2}\frac{D^+}{h_{\bot}^+}+\frac{c(p^-)}{2}
      \frac{D^-}{h_{\bot}^-} & \textrm{ on interior edges, i.e., } e \in
      E_h^i,\\
      & c(p)\frac{D}{h_{\bot}} & \textrm{ on boundary edges, i.e., } e\in
      \partial \mc{D}^n,
    \end{aligned}
    \right.
\end{equation}
where $c(p) =Cp(p+1)$, $C$ is a constant (we used $C=2$), $p$ is the
polynomial order, $D$ is the diffusion coefficient, and $h_{\bot}$ is the
length of the cell in the direction orthogonal to edge $e$.  The + and -
symbols represent the values on either side of an edge. With the penalty
coefficient $\kappa_e^{IP}$, the IP bilinear form is symmetric positive
definite.

\section{Adaptive Mesh Refinement} \label{sec_amr}
In this Section, we introduce Adaptive Mesh Refinement (AMR)
\cite{Jessee1998,Wang2010a,Ragusa2010}. The goal of AMR is to refine the mesh
where it is necessary while keeping the mesh as coarse as possible everywhere
else. It is common to have problems with a quickly varying solution on a small
part of the domain (e.g. boundary layers) and a smooth solution on the rest of 
the domain. AMR allows to create automatically a mesh very fined where the
solution varies quickly while keeping the rest of the mesh coarse. AMR requires 
an \emph{a posteriori} error indicator to decide which cells should be refined. 
In this work, the error indicator on element $K$ is given by \cite{Wang2010a}:
\begin{equation}
  \eta_K = \frac{\int_{\partial K} \ldb\phi_K\rdb^2}{\|\Phi_K\|_2^2},
  \label{error_indic}
\end{equation}  
This error indicator is based on the jump of the scalar flux between
two cells. The larger the jump is the larger the error indicator. Given that
the solution of the diffusion equation is continuous, \cref{error_indic} is a
good error indicator. 

The adaptive mesh refinement used in our code works as follows:
\begin{enumerate}
  \item The solution is computed on a coarse mesh.
  \item The error indicator for each cell is computed.
  \item All the cells $K$ such that $\eta_K > \epsilon \max_{J} \eta_J$ where
    $\epsilon \in [0,1]$ are flagged for refinement.
  \item The flagged cells are refined.
  \item The solution on the coarse mesh is projected on the fine mesh.
  \item Go back to 1.
\end{enumerate}
Step 5 is not necessary but it permits to decrease the time needed to solve
the problem on the finer mesh since a very good initial guess is available to
the solver.

When a discretization that does not allow polygonal cells is employed, the
refinement of the mesh may create hanging nodes. Whereas if polygonal cells can
be used, hanging nodes are not necessary since every time a hanging node
would be needed, one edge is added to 2 cells.

\section{Results} \label{sec_results}
\subsection{Z-mesh}
\subsection{Randomized polygonal grid}
\subsection{AMR}

\input{conclusions}

% bibliography
\bibliographystyle{unsrt}
\bibliography{database}


\end{document}
